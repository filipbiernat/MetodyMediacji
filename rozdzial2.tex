\chapter{Mechanizmy mediacji wiedzy}
\label{cha:mechanizmyMediacjiWiedzy}

%---------------------------------------------------------------------------

\section{Obszary zastosowań}
\label{sec:obszaryZastosowan}

Kluczowymi zastosowaniami mechanizmów mediacji wiedzy są te powiązane z rozpoznawaniem emocji. Przede wszystkim chodzi tu o systemy rekomendacji, które najbardziej mogą skorzystać na dodatkowym rodzaju danych wejściowych, jakimi są emocje użytkownika. 

Wyobraźmy sobie system rekomendacji muzyki dużej aplikacji takiej jak Spotify. Dzięki dodatkowej wiedzy, mógłby zaproponować nam muzykę adekwatną do naszego nastroju. Podobnie sprawa ma się z serwisami filmowymi takimi jak Netfix czy YouTube. Takie serwisy poza wyszukiwarką dysponują też stroną główną, na której przedstawiają użytkownikowi jak najlepsze propozycje. Wykorzystując informacje o naszych emocjach, te propozycje mogłyby być jeszcze lepsze. Kolejne możliwości pojawiają się w obszarze zakupów. Platformy sprzedawcze takie jak Amazon czy Allegro mogłyby skorzystać z efektów mediacji wiedzy proponując bardziej adekwatne produkty. Idąc dalej, przeglądarki internetowe takie jak Google czy DuckDuckGo mogłyby wyświetlać coraz lepsze propozycje.

Innym obszarem, gdzie mogłaby znaleźć zastosowanie wiedza zdobyta dzięki mechanizmom mediacji, mogłyby być gry i materiały edukacyjne. Emocje są niezwykle ważnym elementem przeżywania gier komputerowych. Znając reakcję gracza, system mógłby lepiej dostosować otaczający go świat.

%---------------------------------------------------------------------------

\section{Jawne metody mediacji wiedzy}
\label{sec:jawneMetodyMediacjiWiedzy}

Pierwszą z kategorii metod mediacji wiedzy są metody jawne. Oznacza to, że w proces mediacji zaangażowany jest sam użytkownik. To rozwiązanie ma swoje wady i zalety. Z pewnością największą zaletą jest skuteczność rozwiązania -- nikt nie wie więcej o stanie emocji niż sam użytkownik. Z kolei główną wadą jest fakt, że takie badanie może okazać się uciążliwe i irytujące, a na dłuższą metę zbytnio ingerujące w życie człowieka, czy wręcz niemożliwe.

Przykładem jawnej metody może być praca Emilii Pieczonki realizowana na Katedrze Informatyki Stosowanej. W badaniach, jakie przeprowadziła, ''grupa użytkowników korzystajaca z telefonu Android i aplikacji mobilnej, używała podczas codziennych czynności urządzenia sensorycznego na swoim nadgarstku i odpowiadała na pytania dotyczące ich samopoczucia'' \cite{EmiliaPieczonka}.

Warto również zwrócić uwagę na pracę Arkadiusza Lisa, również na Katedrze Informatyki Stosowanej. W tej pracy zostało wybrane w oparciu o literaturę naukową kilka najbardziej obiecujących sposobów mediacji wiedzy. Następnie zaimplementowano zestaw widgetów, z pomocą których użytkownik ''powinien być w stanie subiektywnie określić stan emocjonalny (...), a następnie przekazać tą informację do systemu'' \cite{ArkadiuszLis}.

W pracy opisane w artykule \cite{hung2016predicting} naukowcy we współpracy z psychiatrą stworzyli system, z którego pomocą badany z wykorzystaniem kolorowego suwaka może określić poziom swojego niepokoju czy gniewu. W pracy badacze skupili się przede wszystkim nie na sensorach, ale na wzorcach korzystania z telefonu udostępniach przez system operacyjnych takich jak wykonywanie połączeń, pisanie SMSów czy lokalizacja.

Mediacja nie musi odbywać się z wykorzystaniem ekranu. Na rynku pojawiają się rozwiązania, w których system głosowy taki jak Alexa, Asystent Google, Cortana czy Siri podczas rozmowy sam wypytuje użytkownika o nastrój.


%---------------------------------------------------------------------------

\section{Niejawne metody mediacji wiedzy}
\label{sec:niejawneMetodyMediacjiWiedzy}


Pieczonka 11 - drunk driving
Pieczonka13-HumanActivityRecognition
Nalepa - w grach


%---------------------------------------------------------------------------

\section{Pozyskiwanie wiedzy o stanie emocjonalnym w systemach \textit{affective computing}}
\label{sec:pozyskiwanieWiedzyOStanieEmocjonalnymWSystemachAffectiveComputing}

Nalepa - w grach
Computer teaching and learning systems - Picard
HAY
HAY photo + w przyszłości sensory
