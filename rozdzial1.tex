\chapter{Wprowadzenie}
\label{cha:wprowadzenie}

%---------------------------------------------------------------------------

\section{Cel pracy}
\label{sec:celPracy}

Celem pracy jest zaprojektowanie, implementacja i ewaluacja mechanizmu pozyskiwania wiedzy o  stanie emocjonalnym użytkownika w systemach \textit{affective computing}.

W szczególności w~ramach pracy przeprowadzona zostanie analiza istniejących rozwiązań, na bazie których opracowane zostało rozwiązanie pozwalające na odpytywanie użytkownika o stan emocjonalny w sposób możliwie nieintruzywny.  Na potrzeby pracy zaprojektowana i zaimplementowana zostanie aplikacja mobilna pozwalająca na uczynienie jawnej mediacji wiedzy możliwie nieintruzywną (poprzez wnioskowanie oparte na monitorowaniu czynników zewnętrznych) oraz na przeprowadzenie niejawnej mediacji (z wykorzystaniem aparatu fotograficznego telefonu) wykorzystując znajomość tzw. kontekstu czyli informacji o otoczeniu użytkownika i dysponując możliwością dynamicznego reagowania na zmiany kontekstu w czasie rzeczywistym.. 

Rozwiązanie zostanie poddane ewaluacji przeprowadzając badanie wśród grona użytkowników, którzy następnie zostaną przepytani, a wyniki kwestionariuszy zostaną opisane w niniejszej pracy.


%---------------------------------------------------------------------------

\section{Zastosowania wykrywania emocji}
\label{sec:zastosowaniaWykrywaniaEmocji}

Nasz świat powoli staje się światem, w którym maszyna potrafi rozpoznawać emocje. Wraz z rozwojem technologii pojawiają się i będą pojawiać kolejne systemy zdolne do wykrywania emocji użytkownika i wykorzystywania ich, aby lepiej spełniać swoje zadania:

\begin{itemize}	
	\item Już teraz pojawiają się pierwsze gry komputerowe wykorzystujące informacje o stanie emocjonalnym użytkownika. Przykładem może być \cite{nalepa2017affective}. Dla aplikacji budujących alternatywny świat, zwłaszcza w coraz popularniejszej wirtualnej czy rozszerzonej rzeczywistości, wiedza o stanie emocjonalnym użytkownika byłaby bezcenna.
	
	\item Na rynku pojawiają się też pierwsze systemy e-learningowe wykorzystujące dodatkowe informacje afektywne do lepszego dopasowania tempa i materiału. Przykładem może być \cite{feidakis2012design}.
	
	\item Praca \cite{hung2016predicting} opisuje zastosowanie wykrywania emocji w medycynie. Takie systemy mogą znaleźć zastosowanie na przykład w walce z depresją, którą już teraz nazywa się chorobą cywilizacyjną dwudziestego pierwszego wieku.
	
	\item Pojawiają się także specjalne systemy afektywne zaprojektowane z myślą o osobach starszych. Oprogramowanie takie jak \cite{yu2014emotion} może wezwać karetkę czy inną pomoc, a także świadczyć inne udogodnienia reagując w czasie rzeczywistym na informacje o stanie emocjonalnym.
	
	\item Naturalnym przykładem zastosowania wykrywania emocji są systemy rekomendacyjne. Platformy multimedialne i streamingowe takie jak Netflix, Spotify, Twitch czy Youtube mogłyby wykorzystywać dodatkowe informacje, aby jeszcze lepiej wybierać muzykę czy filmy.
	
	\item Systemy rekomendacji mają także olbrzymie znaczenie w sprzedaży. Propozycje systemów sprzedających takich jak Amazon, Allegro czy Aliexpress dysponując w przyszłości informacjami o emocjach mogłyby być znacznie dokładniejsze.
\end{itemize}

Część omówionych tu zastosowań to systemy dopiero się pojawiające -- pionierskie w swoich dziedzinach, część ciągle pozostaje w sferze domysłów i hipotez. Jedno jest pewne -- wiedza o stanie afektywnym człowieka byłaby nieocenioną pomocą dla coraz bardziej zaawansowanych systemów w coraz szybciej rozwijającym się świecie. Do rozwiązania pozostaje jeszcze kwestia, w jaki sposób nauczyć komputery emocji. Jedną z propozycji jest gromadzenie dużej ilości danych.

%---------------------------------------------------------------------------

\section{Gromadzenie danych}
\label{sec:gromadzenieDanych}

Telefon komórkowy potrafi odczytać położenie za pomocą żyroskopu, czy tekst za pomocą dotykowej klawiatury. Idea \textit{HowAreYou} -- aplikacji stworzonej na potrzeby niniejsze pracy -- polega na tym, żeby te ograniczone sygnały połączyć i za ich pomocą odczytać nastrój użytkownika. Smartfon może też zrobić zdjęcie kamerą -- za pomocą fotografii już teraz można całkiem dokładnie określić nastrój użytkownika. Można też wprost zapytać użytkownika telefonu –- jeżeli nie nadużyjemy tego rozwiązania, stwarza ono szeroki wachlarz możliwości.

Jednym z najważniejszych wyzwań związanych z projektowaniem aplikacji przetwarzajacych dane użytkwonika jest kwestia prywatności. Człowiek, co naturalne, lubi strzec swojej niezależności. Zrezygnuje z wykorzystywania przygotowanego rozwiązania, jeżeli będzie nieustannie niepokojony, albo poczuje, że traci kontrolę nad danymi, które trafiają do bazy. Konieczne jest zaprojektowanie takiego systemu, który da użytkownikowi możliwość wyboru i pewność, że treści, które udostępnia, są przez niego kontrolowane i całkowicie bezpieczne.


%---------------------------------------------------------------------------

\section{Struktura pracy}
\label{sec:strukturaPracy}

Niniejsza praca składa się z sześciu rozdziałów. Rozdział~\ref{cha:mechanizmyMediacjiWiedzy} stanowi omówienie metod mediacji wiedzy począwszy od zastosowań, poprzez analizę poszczególnych typów metod, aż do przeglądu metod wykorzystywanych w systemach afektywnych. W rozdziale~\ref{cha:architekturaRozwiazania} omówione zostały architektoniczne szczegóły rozwiązania realizowanego w ramach pracy, a w rodziale~\ref{cha:implementacja} szczegóły dotyczące implementacji. Ta część zawiera techniczną analizę problemu oraz opisuje rozwiązania zastosowane podczas realizacji aplikacji mobilnej pozwalającej na uczynienie jawnej mediacji wiedzy możliwie nieintruzywną oraz na przeprowadzenie niejawnej mediacji wykorzystując znajomość tzw. kontekstu. W rozdziale~\ref{cha:ewaluacja} opisane zostało w jaki sposób zweryfikowano, czy przygotowana aplikacja spełnia swoje zadania, a w~rozdziale~\ref{cha:podsumowanie} wymieniono wnioski, jakie pojawiły się podczas badania, i wyliczono sposoby dalszego rozwoju pracy. Dodatek~\ref{cha:instalacjaIPierwszeUruchomienie} zawiera instrukcję instalacji i pierwszego uruchomienia, z kolei dodatek~\ref{cha:wynikiBadaniaNastroju} fragmenty tablic baz danych zgromadzone podczas rzeczywistego badania emocji.
